\documentclass[11pt]{article}
\usepackage[a4paper,margin=1in]{geometry}
\usepackage{amsmath,amssymb}
\usepackage{siunitx}
\usepackage{enumitem}
\usepackage[T1]{fontenc}
\usepackage{lmodern}

\title{Clustered Theory \& Formula Sheet}
\author{}
\date{\today}

\begin{document}
\maketitle

% -------------------------
% Module A
% -------------------------
\section*{\textbf{Module A: Base-Mediated Carbonyl Chemistry (Aldol \& Cannizzaro)}}
\subsection*{\textbf{1) Concept snapshot}}
\begin{itemize}[leftmargin=*]
  \item Aldol reaction: combines two carbonyl units via \(\alpha\)-enolate \(\rightarrow\) \(\beta\)-hydroxy carbonyl (addition) and possibly \(\alpha,\beta\)-unsaturated carbonyl (condensation).
  \item Cannizzaro reaction: disproportionation of \textbf{non-enolizable aldehydes} (no \(\alpha\)-H) in strong base \(\rightarrow\) one alcohol + one carboxylate.
  \item Decide quickly by checking \(\alpha\)-hydrogens: \(\alpha\)-H present \(\Rightarrow\) aldol possible; \(\alpha\)-H absent \(\Rightarrow\) Cannizzaro possible (for aldehydes).
  \item Intramolecular/Polyfunctional cases: multiple \(-CHO\) groups can give mixtures of \(-CH_2OH\) and \(-COO^-\) substitutions on the same aromatic ring.
\end{itemize}

\subsection*{\textbf{2) Formula bank}}
\begin{itemize}[leftmargin=*]
  \item \textbf{Enolate formation (base):} \(\ce{R-CO-CH2-R' + OH^- <=> R-CO-CH^- -R' + H2O}\) (rate/extent depends on acidity and base strength).
  \item \textbf{Aldol addition (general):} \(\ce{Enolate + C=O -> \beta\text{-}hydroxy\ carbonyl}\).
  \item \textbf{Aldol condensation (dehydration):} \(\beta\)-hydroxy carbonyl \(\xrightarrow[\Delta]{\text{base or acid}}\) \(\alpha,\beta\)-unsaturated carbonyl \(+\ce{H2O}\).
  \item \textbf{Cannizzaro (self):} \(2\,\ce{RCHO} + \ce{OH^-} \rightarrow \ce{RCH2OH} + \ce{RCOO^-}\) (then \(\ce{H3O^+}\) workup gives \(\ce{RCOOH}\)).
  \item \textbf{Cannizzaro (cross, key special case):} \(\ce{HCHO}\) is the best hydride donor; with another non-enolizable aldehyde \(\ce{RCHO}\): \(\ce{RCHO -> RCH2OH}\) and \(\ce{HCHO -> HCOO^-}\).
  \item \textbf{Eligibility rule (fast):} Cannizzaro needs aldehyde \(\ce{RCHO}\) with \(\alpha\)-carbon having \(\textbf{no H}\). Ketones generally do \emph{not} Cannizzaro.
\end{itemize}

\subsection*{\textbf{3) Constants \& standard values needed}}
\begin{itemize}[leftmargin=*]
  \item Typical base conditions: concentrated \(\ce{NaOH}\) or \(\ce{KOH}\) for Cannizzaro; dilute base often used for aldol \emph{addition} control.
  \item Common workup: \(\ce{H3O^+}\) converts \(\ce{-COO^-}\) to \(\ce{-COOH}\).
\end{itemize}

\subsection*{\textbf{4) Units \& dimensional checks}}
\begin{itemize}[leftmargin=*]
  \item Not unit-heavy; main trap is \textbf{stoichiometry}: Cannizzaro consumes \(\textbf{2 mol aldehyde}\) per \(\textbf{1 mol alcohol}\) formed (self-reaction).
  \item For cross-Cannizzaro with formaldehyde: count each aldehyde separately; base is catalytic but must be present strongly.
\end{itemize}

\subsection*{\textbf{5) Quick method}}
\begin{enumerate}[leftmargin=*]
  \item Identify carbonyl type(s): aldehyde vs ketone; check for \(\alpha\)-H next to \(-CHO\).
  \item If aldehyde has \(\alpha\)-H: expect aldol/related enolate pathways under base.
  \item If aldehyde has no \(\alpha\)-H: expect Cannizzaro under strong base.
  \item For polyaldehydes: assign some \(-CHO \rightarrow -CH_2OH\) and some \(-CHO \rightarrow -COO^-\) (often on same ring).
  \item Apply workup (\(\ce{H3O^+}\) if present) to convert salts to acids.
\end{enumerate}

\subsection*{\textbf{6) Common mistakes}}
\begin{itemize}[leftmargin=*]
  \item Assuming aldehydes \emph{with} \(\alpha\)-H do Cannizzaro (it is the opposite).
  \item Forgetting that self-Cannizzaro needs \(\textbf{2 equivalents}\) of the aldehyde.
  \item Confusing aldol \(\beta\)-hydroxy product (addition) with dehydrated enone (condensation) without checking \(\Delta\) or conditions.
\end{itemize}

\subsection*{\textbf{7) Mini-visuals (ASCII) if helpful}}
\begin{verbatim}
Aldol (simplified):
  O                 O-
  ||    base        |
R-C-CH2-R'  ->  R-C-CH-R'   (enolate)
                    |
                    attacks another C=O -> beta-hydroxy carbonyl

Cannizzaro (key idea: hydride transfer):
  R-CHO   +   R-CHO   + OH-  ->  R-CH2OH  +  R-COO-
          (one reduced)         (one oxidized)
\end{verbatim}

% -------------------------
% Module B
% -------------------------
\section*{\textbf{Module B: Carbonyl \& Carboxylic Acid Interconversions (Ox/Red, Derivatives, Protecting Groups)}}
\subsection*{\textbf{1) Concept snapshot}}
\begin{itemize}[leftmargin=*]
  \item Oxidation ladder and derivative interconversion enable multi-step synthesis: alkylbenzene \(\rightarrow\) benzoic acid; acid \(\leftrightarrow\) acyl chloride \(\rightarrow\) aldehyde.
  \item Selective reduction to aldehydes: Rosenmund (acyl chloride \(\rightarrow\) aldehyde), DIBAL-H (ester/nitrile \(\rightarrow\) aldehyde under controlled conditions).
  \item Strong oxidants over-oxidize primary alcohols/aldehydes \(\rightarrow\) acids (important trap).
  \item Protecting groups: acetals protect aldehydes/ketones from base/nucleophiles; acid hydrolysis regenerates carbonyl.
  \item Organometallic: Grignard + \(\ce{CO2}\) gives carboxylic acids after workup.
\end{itemize}

\subsection*{\textbf{2) Formula bank}}
\begin{itemize}[leftmargin=*]
  \item \textbf{Benzylic side-chain oxidation:} \(\ce{Ar-CH3/CH2R ->[KMnO4,\ OH^-,\ \Delta][H3O^+] Ar-COOH}\) (requires at least one benzylic H).
  \item \textbf{Acid \(\rightarrow\) acyl chloride:} \(\ce{RCOOH + SOCl2 -> RCOCl + SO2 + HCl}\).
  \item \textbf{Rosenmund reduction:} \(\ce{RCOCl ->[H2,\ Pd/BaSO4] RCHO}\) (poisoned catalyst limits over-reduction).
  \item \textbf{DIBAL-H (controlled):}
    \begin{itemize}[leftmargin=*]
      \item \(\ce{RCOOR' ->[1\ eq\ DIBAL-H,\ -78^{\circ}C][H2O] RCHO}\)
      \item \(\ce{RCN ->[DIBAL-H][H2O] RCHO}\) (careful control; excess/warmer can go further).
    \end{itemize}
  \item \textbf{Jones/Chromic acid oxidation (typical outcomes):}
    \begin{itemize}[leftmargin=*]
      \item \(\ce{1^{\circ}\ alcohol ->[Cr(VI),\ H2SO4] carboxylic\ acid}\)
      \item \(\ce{2^{\circ}\ alcohol ->[Cr(VI)] ketone}\)
      \item \(\ce{aldehyde ->[Cr(VI)] carboxylic\ acid}\)
    \end{itemize}
  \item \textbf{Hydrolysis of derivatives:}
    \begin{itemize}[leftmargin=*]
      \item \(\ce{RCOCl + H2O -> RCOOH + HCl}\) (very fast)
      \item \(\ce{RCOOR' ->[OH^-] RCOO^- + R'OH;\ [H3O^+] RCOOH}\)
      \item \(\ce{RCN ->[H3O^+,\ \Delta] RCOOH}\) or \(\ce{RCN ->[OH^-,\ \Delta] RCOO^-}\)
    \end{itemize}
  \item \textbf{Grignard carboxylation:} \(\ce{RMgX + CO2 -> RCOO^-MgX ->[H3O^+] RCOOH}\).
  \item \textbf{Acetal protection (conceptual):} \(\ce{RCHO + 2ROH <=>[H^+] RCH(OR)2 + H2O}\) (acidic); deprotect \(\ce{RCH(OR)2 ->[H3O^+] RCHO}\).
\end{itemize}

\subsection*{\textbf{3) Constants \& standard values needed}}
\begin{itemize}[leftmargin=*]
  \item Typical oxidant identifiers: \(\ce{KMnO4}\) (strong), \(\ce{H2CrO4}\)/\(\ce{Na2Cr2O7,H2SO4}\) (Jones).
  \item Rosenmund catalyst: \(\ce{Pd/BaSO4}\) (poisoned Pd).
  \item DIBAL-H is \(\ce{(i\text{-}Bu)2AlH}\) (selective at low temperature, \(\sim 1\) equivalent for aldehyde stop).
\end{itemize}

\subsection*{\textbf{4) Units \& dimensional checks}}
\begin{itemize}[leftmargin=*]
  \item Track \textbf{equivalents} (especially DIBAL-H): \(1\) eq at low \(T\) aims for aldehyde; excess often leads to alcohol.
  \item Oxidation state check: aldehyde \(\rightarrow\) acid is a \(2e^{-}\) oxidation; primary alcohol \(\rightarrow\) acid is \(4e^{-}\) overall.
\end{itemize}

\subsection*{\textbf{5) Quick method}}
\begin{enumerate}[leftmargin=*]
  \item Identify functional group: acid, acyl chloride, ester, nitrile, aldehyde, alcohol, protected carbonyl.
  \item Choose the \textbf{mildest} reagent that achieves the target without overreaction.
  \item For aldehyde as target: prefer Rosenmund (from acyl chloride) or DIBAL-H (from ester/nitrile) with tight control.
  \item For benzylic oxidation: ensure benzylic H exists; then \(\ce{KMnO4}\) gives \(\ce{ArCOOH}\).
  \item For Grignard: ensure no protic groups; add \(\ce{CO2}\), then acidic workup.
  \item If an acetal is present: expect stability in base; plan acid deprotection to reveal aldehyde.
\end{enumerate}

\subsection*{\textbf{6) Common mistakes}}
\begin{itemize}[leftmargin=*]
  \item Expecting Jones oxidation to stop at aldehyde for primary alcohols (it usually goes to acid).
  \item Forgetting \(\ce{SOCl2}\) produces gaseous byproducts; reaction driven by loss of \(\ce{SO2}\) and \(\ce{HCl}\).
  \item Using DIBAL-H without temperature/stoichiometry control and accidentally reducing to alcohol.
  \item Missing that \(\ce{KMnO4}\) side-chain oxidation needs at least one benzylic hydrogen.
\end{itemize}

\subsection*{\textbf{7) Mini-visuals (ASCII) if helpful}}
\begin{verbatim}
RCOOH --SOCl2--> RCOCl --H2/Pd-BaSO4--> RCHO
RCOOR' --DIBAL(-78°C,1eq)--> RCHO
RMgX --CO2--> RCOO(-) --H3O+--> RCOOH
\end{verbatim}

% -------------------------
% Module C
% -------------------------
\section*{\textbf{Module C: Key Organic Qualitative Tests \& Reagent Compositions}}
\subsection*{\textbf{1) Concept snapshot}}
\begin{itemize}[leftmargin=*]
  \item Use tests to detect functional groups: carbonyls, alcohols/phenols, C=C, aldehydes vs ketones, methyl ketones.
  \item Many questions reduce to: \textbf{test outcome} \(\Rightarrow\) \textbf{functional group present/absent}.
  \item Learn the \textbf{minimum set}: 2,4-DNP, \(\ce{Na}\) metal, \(\ce{Br2/CCl4}\), neutral \(\ce{FeCl3}\), Tollens, Fehling, iodoform.
\end{itemize}

\subsection*{\textbf{2) Formula bank}}
\begin{itemize}[leftmargin=*]
  \item \textbf{2,4-DNP (Brady's test):} aldehyde/ketone \(\Rightarrow\) yellow/orange precipitate (hydrazone formation).
  \item \textbf{\(\ce{Br2/CCl4}\):} alkene/alkyne \(\Rightarrow\) decolourisation (addition); aromatic ring alone typically does not decolourise without catalyst.
  \item \textbf{Neutral \(\ce{FeCl3}\):} phenols and many enols/\(\beta\)-dicarbonyl enol forms \(\Rightarrow\) coloured complex.
  \item \textbf{Sodium metal test:} alcohols/phenols/acids \(\Rightarrow\) \(\ce{H2}\) gas (effervescence); ethers do not.
  \item \textbf{Tollens' reagent:} \(\ce{[Ag(NH3)2]^+}\) oxidizes aldehydes \(\Rightarrow\) silver mirror; aldehyde \(\rightarrow\) carboxylate.
  \item \textbf{Fehling's test:} aliphatic aldehydes reduce \(\ce{Cu^{2+}}\) \(\Rightarrow\) brick-red \(\ce{Cu2O}\) precipitate (many aromatic aldehydes are negative).
  \item \textbf{Iodoform test:} yellow \(\ce{CHI3}\) precipitate for:
    \begin{itemize}[leftmargin=*]
      \item methyl ketones \(\ce{RCOCH3}\),
      \item secondary alcohols of type \(\ce{R-CH(OH)-CH3}\) (oxidize to methyl ketone),
      \item ethanol/ethanal (special case).
    \end{itemize}
\end{itemize}

\subsection*{\textbf{3) Constants \& standard values needed}}
\begin{itemize}[leftmargin=*]
  \item \textbf{Fehling's reagent composition:} aqueous \(\ce{CuSO4}\) + \textbf{alkaline} sodium potassium tartrate (Rochelle salt) (with \(\ce{NaOH/KOH}\)).
  \item \textbf{Tollens' reagent:} freshly prepared ammoniacal \(\ce{AgNO3}\) giving \(\ce{[Ag(NH3)2]^+}\).
  \item \textbf{Iodoform conditions:} \(\ce{I2/NaOH}\) (or hypoiodite \(\ce{NaOI}\)).
\end{itemize}

\subsection*{\textbf{4) Units \& dimensional checks}}
\begin{itemize}[leftmargin=*]
  \item Not unit-heavy; main trap is \textbf{false negatives}: aromatic aldehydes often negative with Fehling but positive with Tollens.
  \item Different solvents: \(\ce{Br2/CCl4}\) is for unsaturation in non-aqueous medium; aqueous bromine behaves differently.
\end{itemize}

\subsection*{\textbf{5) Quick method}}
\begin{enumerate}[leftmargin=*]
  \item If 2,4-DNP positive \(\Rightarrow\) carbonyl (aldehyde/ketone) present.
  \item If \(\ce{Na}\) metal reacts \(\Rightarrow\) protic group (alcohol/phenol/acid); if no reaction \(\Rightarrow\) likely ether/carbonyl without \(-OH\).
  \item If \(\ce{Br2/CCl4}\) decolourises \(\Rightarrow\) C=C/C\(\equiv\)C present.
  \item If neutral \(\ce{FeCl3}\) gives colour \(\Rightarrow\) phenolic/enolic \(-OH\) (often resonance-stabilized).
  \item For aldehyde confirmation: Tollens \(>\) Fehling for aromatic cases.
  \item For methyl-ketone/related alcohol: iodoform test.
\end{enumerate}

\subsection*{\textbf{6) Common mistakes}}
\begin{itemize}[leftmargin=*]
  \item Treating Fehling as a universal aldehyde test (it is not).
  \item Assuming 2,4-DNP detects acids/esters (it primarily detects aldehydes/ketones).
  \item Forgetting iodoform test excludes \(\ce{1^{\circ}}\) alcohols like \(\ce{CH3CH2CH2OH}\) (unless it is ethanol).
\end{itemize}

\subsection*{\textbf{7) Mini-visuals (ASCII) if helpful}}
\begin{verbatim}
Iodoform "motif":
  R-CO-CH3   or   R-CH(OH)-CH3  -->  CHI3 (yellow ppt)

Fehling:
  Cu2+ (blue)  -->  Cu2O (brick red)
\end{verbatim}

% -------------------------
% Module D
% -------------------------
\section*{\textbf{Module D: Acidity Trends (Carboxylic Acids, Phenols, Hydrohalic \& Oxyacids)}}
\subsection*{\textbf{1) Concept snapshot}}
\begin{itemize}[leftmargin=*]
  \item Acid strength increases when the conjugate base is stabilized (by \(-I\), resonance, high electronegativity, oxidation state effects, and solvation).
  \item \textbf{Carboxylic acids} are generally stronger than \textbf{phenols} due to resonance-stabilized carboxylate.
  \item \(-I\) groups (e.g., \(\ce{Cl}\), \(\ce{NO2}\)) increase acidity; effect \textbf{decreases with distance}.
  \item In hydrohalic acids: bond strength and size dominate in water: \(\ce{HF < HCl < HBr < HI}\).
  \item In oxyacids of same central atom: more oxygens (higher oxidation state) \(\Rightarrow\) stronger acid: \(\ce{HOCl < HClO2 < HClO3 < HClO4}\).
\end{itemize}

\subsection*{\textbf{2) Formula bank}}
\begin{itemize}[leftmargin=*]
  \item \textbf{Definition:} \(K_a = \dfrac{[\ce{H3O^+}][\ce{A^-}]}{[\ce{HA}]}\), \(pK_a = -\log K_a\) (lower \(pK_a\) \(\Rightarrow\) stronger acid).
  \item \textbf{Inductive distance rule (qualitative):} \(-I\) effect \(\propto \dfrac{1}{\text{distance}}\) along \(\sigma\)-bonds (rapidly decays after a few carbons).
  \item \textbf{Resonance stabilization:}
    \begin{itemize}[leftmargin=*]
      \item Carboxylate: charge delocalized over two O atoms.
      \item Phenoxide: charge delocalized into ring but less stabilized than carboxylate.
    \end{itemize}
  \item \textbf{Ortho effect (often used qualitatively):} ortho-substituted benzoic acids can show higher acidity due to field/steric/solvation effects; for exam-level ordering, prioritize strong \(-I/-M\) and proximity.
\end{itemize}

\subsection*{\textbf{3) Constants \& standard values needed}}
\begin{itemize}[leftmargin=*]
  \item Typical \(pK_a\) (aqueous, approximate): formic \(\sim 3.75\), acetic \(\sim 4.76\), chloroacetic \(\sim 2.9\), benzoic \(\sim 4.2\), phenol \(\sim 10\).
  \item Hydrohalic trend in water: \(\ce{HF}\) weakest, \(\ce{HI}\) strongest.
  \item Oxyacid trend (chlorine series): \(\ce{HOCl < HClO2 < HClO3 < HClO4}\).
\end{itemize}

\subsection*{\textbf{4) Units \& dimensional checks}}
\begin{itemize}[leftmargin=*]
  \item \(K_a\) is dimensionless in standard-state convention; \(pK_a\) has no units.
  \item When comparing, ensure you are comparing under the same solvent conditions (implicit: aqueous).
\end{itemize}

\subsection*{\textbf{5) Quick method}}
\begin{enumerate}[leftmargin=*]
  \item Identify the acidic proton source (carboxylic acid vs phenol vs mineral oxyacid vs hydrohalic).
  \item Stabilization check: add resonance contributors and assess \(-I\)/\(-M\) groups.
  \item For substituted carboxylic acids: closer halogen/\(\ce{NO2}\) \(\Rightarrow\) stronger acid; branching near \(\ce{-COOH}\) can donate by \(+I\) and weaken acidity.
  \item For hydrohalic acids: larger halide \(\Rightarrow\) stronger (bond weaker, anion stabilized).
  \item For oxyacids: more \(=O\) groups \(\Rightarrow\) stronger (greater \(-I\) and conjugate base delocalization).
\end{enumerate}

\subsection*{\textbf{6) Common mistakes}}
\begin{itemize}[leftmargin=*]
  \item Forgetting inductive effect decays with distance (e.g., \(\beta\)- vs \(\gamma\)-chloro acids).
  \item Treating phenols as stronger than carboxylic acids (generally false).
  \item Reversing hydrohalic trend by electronegativity alone (aqueous acidity follows \(\ce{HF < HCl < HBr < HI}\)).
\end{itemize}

\subsection*{\textbf{7) Mini-visuals (ASCII) if helpful}}
\begin{verbatim}
Distance effect:
  Cl-CH2-COOH (stronger)  >  Cl-CH2-CH2-COOH (weaker)  >  Cl-CH2-CH2-CH2-COOH (even weaker)

Oxyacids:
  HO-Cl(=O)n : increasing n increases acidity
\end{verbatim}

% -------------------------
% Module E
% -------------------------
\section*{\textbf{Module E: Aromatic Transformations (EAS, Reductions, Diazotization, Acylation, Bromination)}}
\subsection*{\textbf{1) Concept snapshot}}
\begin{itemize}[leftmargin=*]
  \item Electrophilic aromatic substitution (EAS) governs nitration/sulfonation patterns via directing effects.
  \item Nitro \(\rightarrow\) amine reduction enables downstream acylation (amide formation) and high activation for bromination.
  \item Diazotization of aromatic amines gives diazonium salts, enabling substitution/hydrolysis to phenols.
  \item Sulfonation can act as a temporary blocking group (introduce then remove).
\end{itemize}

\subsection*{\textbf{2) Formula bank}}
\begin{itemize}[leftmargin=*]
  \item \textbf{Nitration:} \(\ce{Ar-H ->[conc.\ HNO3/conc.\ H2SO4] Ar-NO2}\) (electrophile \(\ce{NO2^+}\)).
  \item \textbf{Sulfonation:} \(\ce{Ar-H <=>[conc.\ H2SO4,\ \Delta] Ar-SO3H}\) (reversible; can be removed with hot dilute acid/water depending on conditions).
  \item \textbf{Nitro reduction (common):} \(\ce{Ar-NO2 ->[Sn/HCl\ or\ H2/Pd] Ar-NH2}\).
  \item \textbf{Acylation of amine (protect/derivatize):} \(\ce{Ar-NH2 + (CH3CO)2O -> Ar-NHCOCH3 + CH3COOH}\) or \(\ce{CH3COCl}\) similarly.
  \item \textbf{Diazotization \& hydrolysis:}
    \[
      \ce{Ar-NH2 ->[NaNO2/HCl,\ 0{-}5^{\circ}C] Ar-N2^+Cl^- ->[H2O,\ \Delta] Ar-OH}
    \]
  \item \textbf{Bromination of strongly activated rings (e.g., anilines):} often gives polybromination; controlling conditions (solvent/acidity) matters.
  \item \textbf{Directing effects (core):}
    \begin{itemize}[leftmargin=*]
      \item Activating \(o/p\): \(\ce{-OH}\), \(\ce{-OR}\), \(\ce{-NH2}\), \(\ce{-NHR}\), \(\ce{-NR2}\).
      \item Deactivating meta: \(\ce{-CHO}\), \(\ce{-COR}\), \(\ce{-COOH}\), \(\ce{-COOR}\), \(\ce{-SO3H}\), \(\ce{-NO2}\), \(\ce{-CN}\).
    \end{itemize}
\end{itemize}

\subsection*{\textbf{3) Constants \& standard values needed}}
\begin{itemize}[leftmargin=*]
  \item Diazotization temperature window: \(\sim 0{-}5^{\circ}\text{C}\) (stability of diazonium salts).
  \item Nitration mixture: conc. \(\ce{HNO3/H2SO4}\) generates \(\ce{NO2^+}\).
\end{itemize}

\subsection*{\textbf{4) Units \& dimensional checks}}
\begin{itemize}[leftmargin=*]
  \item Temperature is a control parameter (especially diazotization); track in \(^\circ\text{C}\) or K consistently.
  \item In multi-step aromatic sequences, atom count checks (C/H/N/O) help verify substitutions vs oxidations.
\end{itemize}

\subsection*{\textbf{5) Quick method}}
\begin{enumerate}[leftmargin=*]
  \item Mark current ring substituents and their directing effects (activate/deactivate; \(o/p\) vs meta).
  \item For sequences, apply transformations one-by-one: reduce \(\ce{-NO2}\) to \(\ce{-NH2}\), then optionally acylate to control activation.
  \item Use sulfonation as a reversible block if a later nitration needs positional control.
  \item If diazonium appears: decide the fate (hydrolysis to phenol, substitution, etc.).
  \item For bromination: anticipate multiple substitution on highly activated rings unless protected/deactivated.
\end{enumerate}

\subsection*{\textbf{6) Common mistakes}}
\begin{itemize}[leftmargin=*]
  \item Forgetting \(\ce{-CO-}\) substituents are meta-directing (e.g., acetyl group on ring).
  \item Attempting diazotization at warm temperatures (decomposition/side products).
  \item Ignoring over-activation of aniline (often leads to rapid polybromination).
\end{itemize}

\subsection*{\textbf{7) Mini-visuals (ASCII) if helpful}}
\begin{verbatim}
Meta-director example:
  Ar-COCH3  directs incoming E+ to meta positions.

Diazotization route:
  Ar-NH2 -> Ar-N2+ -> Ar-OH (on hydrolysis)
\end{verbatim}

% -------------------------
% Module F
% -------------------------
\section*{\textbf{Module F: Alkyne/Alkene Transformations (Hydration, Reduction, Dehydration, Ozonolysis)}}
\subsection*{\textbf{1) Concept snapshot}}
\begin{itemize}[leftmargin=*]
  \item Mercury-catalyzed hydration of alkynes gives enol \(\rightarrow\) ketone (tautomerization).
  \item \(\ce{NaBH4}\) reduces aldehydes/ketones to alcohols (generally not acids/esters under standard conditions).
  \item Dehydration of alcohols (acid-catalyzed) gives alkenes; ozonolysis cleaves alkenes to carbonyl fragments.
  \item Symmetry/product identity checks: ozonolysis yielding only acetone implies a highly symmetric alkene precursor.
\end{itemize}

\subsection*{\textbf{2) Formula bank}}
\begin{itemize}[leftmargin=*]
  \item \textbf{Alkyne hydration (Markovnikov):} \(\ce{RC#CH ->[HgSO4,\ dil.\ H2SO4] R-CO-CH3}\) (terminal alkyne \(\rightarrow\) methyl ketone).
  \item \textbf{Keto--enol tautomerism:} \(\ce{enol <=> ketone}\) (ketone usually favored).
  \item \textbf{Carbonyl reduction:} \(\ce{R2C=O ->[NaBH4] R2CHOH}\).
  \item \textbf{Alcohol dehydration:} \(\ce{R-CH2-CH(OH)-R' ->[conc.\ H2SO4,\ \Delta] R-CH=CH-R' + H2O}\).
  \item \textbf{Ozonolysis (reductive workup):} \(\ce{RCH=CHR' ->[O3][Zn/H2O] RCHO + R'CHO}\) (or ketones depending on substitution).
\end{itemize}

\subsection*{\textbf{3) Constants \& standard values needed}}
\begin{itemize}[leftmargin=*]
  \item Key reagents: \(\ce{HgSO4/dil.\ H2SO4}\) (hydration), \(\ce{NaBH4}\) (reduction), \(\ce{O3}\) then \(\ce{Zn/H2O}\) (reductive ozonolysis).
\end{itemize}

\subsection*{\textbf{4) Units \& dimensional checks}}
\begin{itemize}[leftmargin=*]
  \item Ensure correct molecular formula tracking when \(\ce{H2O}\) adds (hydration) and when \(\ce{H2O}\) leaves (dehydration).
  \item Ozonolysis carbon balance: total carbons in products must equal carbons in alkene.
\end{itemize}

\subsection*{\textbf{5) Quick method}}
\begin{enumerate}[leftmargin=*]
  \item For \(\ce{HgSO4/H2SO4}\) on alkynes: write Markovnikov enol then tautomerize to ketone.
  \item Apply \(\ce{NaBH4}\): convert \(>\!C{=}O\) to \(>\!C{-}OH\).
  \item Dehydrate alcohol: form the most substituted alkene consistent with mechanism (E1/E2-like under acid).
  \item Ozonolysis: cut the double bond; assign each alkene carbon to a carbonyl carbon.
  \item If only one carbonyl type appears, infer symmetry in the alkene.
\end{enumerate}

\subsection*{\textbf{6) Common mistakes}}
\begin{itemize}[leftmargin=*]
  \item Forgetting enol \(\rightarrow\) ketone tautomerization after hydration.
  \item Treating \(\ce{NaBH4}\) as a reducer of carboxylic acids/esters under normal conditions.
  \item Miscounting ozonolysis fragments (carbon balance error).
\end{itemize}

\subsection*{\textbf{7) Mini-visuals (ASCII) if helpful}}
\begin{verbatim}
Ozonolysis "cut":
  R1R2C=CR3R4  -->  R1R2C=O  +  O=CR3R4
\end{verbatim}

% -------------------------
% Module G
% -------------------------
\section*{\textbf{Module G: IUPAC Naming of Polycarboxylic Acids}}
\subsection*{\textbf{1) Concept snapshot}}
\begin{itemize}[leftmargin=*]
  \item Carboxylic acid has high priority; multiple \(\ce{-COOH}\) groups use suffixes like \textbf{dioic acid}, \textbf{tricarboxylic acid}, etc.
  \item Choose the parent chain that includes the \textbf{maximum number of \(\ce{-COOH}\)} groups; then number to give lowest locants.
  \item Additional \(\ce{-COOH}\) groups not in the main chain are named as \textbf{carboxy-} substituents.
\end{itemize}

\subsection*{\textbf{2) Formula bank}}
\begin{itemize}[leftmargin=*]
  \item \textbf{Suffix patterns:}
    \begin{itemize}[leftmargin=*]
      \item Two acids on chain ends: \(\text{alkane-}1,n\text{-dioic acid}\).
      \item Three acids on same skeleton: \(\text{alkanetricarboxylic acid}\) or \(\text{carboxy-}\) substituted dioic acid (depending on chosen parent).
    \end{itemize}
  \item \textbf{Numbering rule:} assign numbers so that principal functional group locants are minimized first; then substituents.
\end{itemize}

\subsection*{\textbf{3) Constants \& standard values needed}}
\begin{itemize}[leftmargin=*]
  \item Root chain names: methane, ethane, propane, butane, pentane, hexane, \dots
  \item Multipliers: di-, tri-, tetra-.
\end{itemize}

\subsection*{\textbf{4) Units \& dimensional checks}}
\begin{itemize}[leftmargin=*]
  \item Not unit-based; correctness check is structural: carbon count in parent + substituents must match drawn structure.
\end{itemize}

\subsection*{\textbf{5) Quick method}}
\begin{enumerate}[leftmargin=*]
  \item Identify all \(\ce{-COOH}\) groups and decide whether they lie on a single continuous chain.
  \item Pick the parent chain containing the maximum \(\ce{-COOH}\) groups.
  \item Number the chain to give the lowest locants to the acid groups.
  \item Name remaining \(\ce{-COOH}\) as \textbf{carboxy-} substituents with locants.
  \item Assemble full name with commas and hyphens correctly.
\end{enumerate}

\subsection*{\textbf{6) Common mistakes}}
\begin{itemize}[leftmargin=*]
  \item Choosing a longer carbon chain that contains fewer \(\ce{-COOH}\) groups (wrong priority).
  \item Mislabeling extra \(\ce{-COOH}\) as \(\text{-oic acid}\) when it must be \textbf{carboxy-} substituent.
\end{itemize}

\subsection*{\textbf{7) Mini-visuals (ASCII) if helpful}}
\begin{verbatim}
If HOOC-CH2-CH(-COOH)-CH2-CH2-COOH:
Pick chain including two COOH as dioic acid backbone,
the extra COOH becomes "carboxy-" substituent.
\end{verbatim}

% -------------------------
% Module H
% -------------------------
\section*{\textbf{Module H: Solvolysis (SN1) of Benzylic/Tertiary Halides in Aqueous Media}}
\subsection*{\textbf{1) Concept snapshot}}
\begin{itemize}[leftmargin=*]
  \item In polar protic/aqueous solvents, tertiary/benzylic halides often undergo \textbf{SN1} via carbocations.
  \item Product mixtures arise from rearrangements, competing nucleophiles (water/acetone), and stereochemical scrambling.
  \item Hydrolysis in aqueous acetone typically means \textbf{solvolysis}: solvent participates; rate depends on carbocation stability.
\end{itemize}

\subsection*{\textbf{2) Formula bank}}
\begin{itemize}[leftmargin=*]
  \item \textbf{SN1 rate law:} \(\text{rate} = k[\text{substrate}]\) (leaving group departure is rate-determining).
  \item \textbf{Carbocation stability (common order):} benzylic/allylic \(\gtrsim\) \(3^\circ > 2^\circ \gg 1^\circ\).
  \item \textbf{Outcomes:}
    \begin{itemize}[leftmargin=*]
      \item Substitution: \(\ce{R-X -> R-OH}\) (water capture), then deprotonation.
      \item Rearrangement: hydride/alkyl shifts \(\Rightarrow\) more stable carbocation \(\Rightarrow\) different alcohol skeleton.
      \item Stereochemistry: planar carbocation \(\Rightarrow\) racemization (if chiral center forms).
    \end{itemize}
\end{itemize}

\subsection*{\textbf{3) Constants \& standard values needed}}
\begin{itemize}[leftmargin=*]
  \item Solvent polarity: aqueous mixtures stabilize ions, accelerating SN1.
  \item Leaving group ability (typical): \(\ce{I^- > Br^- > Cl^- \gg F^-}\).
\end{itemize}

\subsection*{\textbf{4) Units \& dimensional checks}}
\begin{itemize}[leftmargin=*]
  \item Kinetics: \(k\) has units of \(\text{s}^{-1}\) for first-order reactions.
  \item Structural accounting: rearrangements preserve molecular formula; only connectivity changes.
\end{itemize}

\subsection*{\textbf{5) Quick method}}
\begin{enumerate}[leftmargin=*]
  \item Identify if substrate can form a stable carbocation (tertiary/benzylic/allylic).
  \item Write the carbocation after leaving group departure; check for adjacent rearrangement options.
  \item Add water as nucleophile to carbocation; then deprotonate to alcohol.
  \item If multiple cations feasible, list corresponding alcohol products; use stability to decide major/minor.
\end{enumerate}

\subsection*{\textbf{6) Common mistakes}}
\begin{itemize}[leftmargin=*]
  \item Forcing SN2 on sterically hindered/benzylic tertiary centers in protic solvent (SN1 dominates).
  \item Ignoring rearrangements when a more stable carbocation is one shift away.
  \item Assuming stereospecificity (SN1 is not stereospecific).
\end{itemize}

\subsection*{\textbf{7) Mini-visuals (ASCII) if helpful}}
\begin{verbatim}
SN1:
  R-X  ->  R+  + X-
  R+ + H2O -> R-OH2+ -> R-OH

Rearrangement:
  less stable R+  --shift-->  more stable R+
\end{verbatim}

% -------------------------
% Module I
% -------------------------
\section*{\textbf{Module I: Multi-Step Stoichiometry with Percent Yields}}
\subsection*{\textbf{1) Concept snapshot}}
\begin{itemize}[leftmargin=*]
  \item Many synthetic chains ask for final mass given starting moles and stepwise yields.
  \item Treat each step as a conversion factor: moles propagate multiplicatively with yield fractions.
  \item Keep stoichiometric coefficients explicit (often \(1{:}1\), but haloform-type steps create side products).
\end{itemize}

\subsection*{\textbf{2) Formula bank}}
\begin{itemize}[leftmargin=*]
  \item \textbf{Yield propagation:}
  \[
    n_{\text{final}} = n_0 \times \prod_i \left(\frac{Y_i}{100}\right)\times \left(\text{stoichiometric ratio}\right)
  \]
  \item \textbf{Mass from moles:} \(m = n \times M\), where \(M\) is molar mass in \(\si{g\,mol^{-1}}\).
  \item \textbf{Molar mass from atomic weights:} \(M = \sum (\text{atoms})\times(\text{atomic weight})\).
\end{itemize}

\subsection*{\textbf{3) Constants \& standard values needed}}
\begin{itemize}[leftmargin=*]
  \item Common atomic weights used in problems (if specified): \(\ce{H}=1\), \(\ce{C}=12\), \(\ce{N}=14\), \(\ce{O}=16\), \(\ce{Br}=80\) (all in \(\si{g\,mol^{-1}}\)).
  \item Percent to fraction: \(60\% = 0.60\), etc.
\end{itemize}

\subsection*{\textbf{4) Units \& dimensional checks}}
\begin{itemize}[leftmargin=*]
  \item Yields are unitless fractions; moles remain moles stepwise.
  \item Final mass in grams requires \(M\) in \(\si{g\,mol^{-1}}\).
\end{itemize}

\subsection*{\textbf{5) Quick method}}
\begin{enumerate}[leftmargin=*]
  \item Write starting moles \(n_0\).
  \item For each step, multiply by yield fraction \(\left(\frac{Y}{100}\right)\).
  \item Apply any stoichiometric factor if products are not formed \(1{:}1\).
  \item Compute molar mass \(M\) using the given atomic weights.
  \item Convert to mass \(m=nM\).
\end{enumerate}

\subsection*{\textbf{6) Common mistakes}}
\begin{itemize}[leftmargin=*]
  \item Applying yields to mass instead of moles mid-chain without consistency.
  \item Forgetting that yields apply sequentially (you do not average them).
  \item Molar mass errors from missing halogens (e.g., bromine contribution is huge).
\end{itemize}

\subsection*{\textbf{7) Mini-visuals (ASCII) if helpful}}
\begin{verbatim}
n0 --(Y1)--> n1 --(Y2)--> n2 --(Y3)--> n3
n3 * M = final grams
\end{verbatim}

% -------------------------
% Appendix: Question–Module Mapping
% -------------------------
\section*{\textbf{Appendix: Question–Module Mapping}}
\begin{center}
\begin{tabular}{|c|c|}
\hline
\textbf{Question No.} & \textbf{Covered Module(s)} \\
\hline
1 & Module A \\
2 & Module C \\
3 & Module E \\
4 & Module D \\
5 & Module A, Module B \\
6 & Module B \\
7 & Module G \\
8 & Module C \\
9 & Module F \\
10 & Module D \\
11 & Module B, Module C, Module E \\
12 & Module B, Module E \\
13 & Module A \\
14 & Module C \\
15 & Module B \\
16 & Module D \\
17 & Module B, Module C, Module E, Module I \\
18 & Module A \\
19 & Module A \\
20 & Module H \\
21 & Module B \\
22 & Module A \\
23 & Module C \\
24 & Module A \\
25 & Module A, Module B, Module C, Module E \\
\hline
\end{tabular}
\end{center}

\end{document}
